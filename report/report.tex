\documentclass{acm_proc_article-sp}

\usepackage[T1]{fontenc}
\usepackage{polyglossia}
\setdefaultlanguage{english}
\usepackage{csquotes}

\usepackage{fontspec}
\usepackage{xltxtra}
\usepackage{libertine}

\usepackage[usenames, dvipsnames]{xcolor}
\graphicspath{{./img/}}

\usepackage[backend=biber, style=alphabetic]{biblatex}\bibliography{literatur.bib}

\usepackage{subcaption}
\usepackage{fancyref}

\usepackage[%
unicode=true,%
colorlinks=true,%
linkcolor=black,%
urlcolor=MidnightBlue,%
citecolor=black,%
filecolor=black%
]
{hyperref}


\newcommand{\todo}[1]{\textcolor{Red}{#1}}
\newcommand{\sebastian}[1]{\textcolor{Green}{#1}}
\newcommand{\stefan}[1]{\textcolor{BurntOrange}{#1}}
\newcommand{\etal}{\textit{et. al.}}
\newcommand{\Gray}[1]{\textcolor{Gray}{#1}}

\begin{document}

\title{
Interactive Simulation WS 15/16\\ %
Project Report
}
\subtitle{EYES - Exchange Your Vision Simulator}
\numberofauthors{2}
\author{
% 1st. author
\alignauthor
Sebastian Lemp\\
%       \affaddr{Street, House}\\
%       \affaddr{PLZ City}\\
%       \affaddr{Country}\\
%       \email{sebastian.lemp@student.uni-augsburg.de}
% 2nd. author
\alignauthor
Stefan Büttner\\
%       \affaddr{Street, House}\\
%       \affaddr{PLZ City}\\
%       \affaddr{Country}\\
%       \email{stefan.buettner@student.uni-augsburg.de}
}
%\additionalauthors{Additional Authors}

% The date is actually not used in the acm template
\date{University of Augsburg, \today}

% Not neede for our purposes
%\terms{Terms}
%\keywords{Keyword 1, Keyword 2}
%% A category with the (minimum) three required fields
%\category{H.4}{Information Systems Applications}{Miscellaneous}
%%A category including the fourth, optional field follows...
%\category{D.2.8}{Software Engineering}{Metrics}[complexity measures, performance measures]

%% For the ACM ToG format
%\acmformat{ACMFormat}
%\acmVolume{Vol.}
%\acmNumber{Nr.}
%\acmYear{YYYY}
%\acmMonth{MM}
%\acmArticleNum{XXX}
%\doi{DOI}


\maketitle
%\begin{abstract}
%\end{abstract}

% Disease list:
% -------------------------------------------------------------------------------
% (Stefan)     11 Disorders of sclera, cornea, iris and ciliary body
% (Sebastian)   1 Cataract (Grauer Star)
% (Stefan)      2 Retinal detachment and breaks
% (Sebastian)  14 Other retinal disorders
% (Stefan)      1 Glaucoma (Grüner Star)
% (Stefan)      2 Disorders of optic nerve and visual pathways
% (Sebastian)  10 Disorders of ocular muscles, binocular movement, accommodation and refraction
% (Stefan)      6 Visual disturbances and blindness
%              47

%
% Possible References:
% http://www.svi.cps.utexas.edu/EI466209.pdf

% http://www.icdvrat.org/2008/papers/ICDVRAT2008_S04_N06_Banks_McCrindle.pdf
%
% claim they have an real-time app for Android and iOS:
% http://www.brailleinstitute.org/sight-loss-blog/398-leading-eye-diseases.html 
%
% OpenGL real-time simulation
% http://percept.eecs.yorku.ca/papers/p127-vinnikov.pdf
% 

\section{Motivation}
Eye diseases have been an issue throughout all the human history. In the
beginning the focus layed on their treatment. This task is reasonably well
solved for many diseases and since Virtual Reality (VR) is more and more
pushing into the consumer market, it could be broadly used for education and
thus prevention of eye diseases. For example, the risk of suffering from
retinal detachment can be greatly reduced if the signs are recognized early
and a doctor is consulted. Therefore, educational software can be used.
In addition, people would hopefully visit a doctor earlier if they already
experienced a good simulation of a severe state of a disease, before they
actually are in a severe state.
Other applications could be testing designs of consumer products like packaging
or traffic signs or other signs at public places.

Although there are many simulations available already, they usually work on
still 2D images, 2D video streams or static 3D scenes\footnote{All these
statements are based on our brief research. There might still be game-like
applications out there... somewhere.} and don't have any game component.
Moreover, more sophisticated simulations are probably not easily available
for public use and implementing a simulator using Unity3D in terms of an
\emph{eye disease asset set} wrapped into a small game could be interesting
for a broad audience.

%\begin{itemize}
%  \item Give people the opportunity to experience different symptoms of eye
%      disease, to know what they are and how to fight them for example
%      nightblindness can caused by a wrong diet
%      ⇒ so help people to eat the right food.
%  \item Supposedly there is already a real-time simulator for eye diseases for
%      Android, although we were not able to find it on Google Play Store for
%      our phones \cite{braille}.
%\end{itemize}

\section{Concept}
%\begin{itemize}
%  \item Fulfill everyday tasks with impaired vision
%  \item Possible boni: 
%  \begin{itemize}
%    \item Get rid of disease by performing the right actions
%    e.g. take medication on the way or make a doctors appointment...
%    \item Preventive measures during the task to not get disease in next lvl
%  \end{itemize}
%  \item Target platform: Android \& Google Cardboard to address many people
%\end{itemize}

The user should be able to experience different types of eye diseases,
including early as well as severe stages in order to understand when to consult
a doctor and why. Hopefully, this gives people better judgment on when to
visit a doctor as well as the courage to do so, if they experience the symptoms
of a particular disease.

There will be at least the first 6 different eye diseases in
\Fref{tab:eye_diseases} either to choose from in every task or appear at least
once in the game.

Possible tasks could be based on reading (visual acuity), distinguishing
objects (color), and navigating in everyday environments
(limited field of view).
This may be realized required for cooking, working at a line in a factory
sorting screws or similar things, using public transport or driving a car
(at night), food shopping, going to the pharmacy getting the right medication,
find hidden object given written hints, or board/care games.

Ideally the chosen tasks would be individual levels which depend on each other
and will tell a small story, like car driving → food shopping → cooking.
The user can either choose the disease for the level himself or a random disease
is selected in the beginning.
The chosen disease should become worse over the time (within one level or
across levels) but, if possible and accurate (neglecting the time), the user
should also be able to slow down the process or even heal the disease
completely. Therefore he/she has to take the appropriate measures for the
specific disease.

\section{Technical Aspects}
In order to convincingly convey the topic, using a VR device like the
Oculus Rift or an mobile phone / Google Cardboard combination would
be beneficial to the project.

Vinnikov \etal \cite{gazedisplays} developed a Gaze-Contingent-Display in order
to evaluate the users eye direction and adept the displayed images in real-time.
Because the effects of eye diseases follow the eye movement, i.e. are static
with respect to the eye coordinate frame, they achieved more realistic results
in comparison to rendering gaze-independent images.
A consumer solution is under development by the German company
SensoMotoric Instruments (SIM) which provides an gaze tracking solution update
for the Oculus Rift DK 2 \cite{smi-oculus, arstechoculus}.
According to their website they also provide an integration into various VR
engines, including Unity3D, available making it especially interesting for
this project.

The gaze-direction would be useful to accurately simulate the vision fields
and would also be an interesting human interface for the game mechanics.

As described in \cite{gazedisplays} and \cite{eyediseasesim} effects like
blurry or distorted vision, floaters, and reduced field of view can be
efficiently implemented by using fragment shaders. The individual properties
of the shaders and how to decompose the individual diseases into different
shaders (re-usability) is subject to the first research block. But since this
appears to be a very well researched topic, we're confident that we won't run
into any major difficulties.

\begin{figure}
    \centering
    \includegraphics[width=\columnwidth]{human_eye_scheme.pdf}
    \caption{Scheme of the human eye.}
    \label{humaneye}
\end{figure}

\begin{table}
    \textbf{Glaucoma}\\
    Sudden eye pain, blurred vision and loss of vision especially in the
    outer regions.

    \vspace{1em}\textbf{Cataracts}\\
    Blurred vision especially in the center region.

    \vspace{1em}\textbf{Diabetic Retinopathy}\\
    Black spots in the view.

    \vspace{1em}\textbf{Color blindness}\\
    Some colours appear indistinguishable.

    \vspace{1em}\textbf{Achromatopsia}\\
    (Almost) No color sensitivity at all.

    \vspace{1em}\textbf{Myopia / Hyperopia}\\
    Commonly known as nearsightedness and farsightedness respectively.

    \vspace{1em}\textbf{Kreatoconus}\\
    The cornea deforms into a conical shape.
    Multiple ghost images may be visible, arranged in a chaotic pattern,
    the vision becomes blurry, and visual acuity decreases at all distances.
    Poor night vision, photo-phobia, and eye strain are additional symptoms.

    \vspace{1em}\textbf{Nyctalopia / Hermalopia}\\
    High difficulty to see in relatively low and bright light respectively.

    \vspace{1em}\textbf{Retinal detachment / Posterior vitreous detachment}\\
    Flashes of light, very brief in the extreme peripheral region.
    Sudden increase in the amount of floaters.
    Slight feeling of heaviness in the eye.
    \caption{Eye diseases}
    \label{tab:eye_diseases}
\end{table}

\newpage
\section{Time-line}
The dates behind the categories are the due dates.
\begin{itemize}

  \item Research \Gray{- 15. Nov. '15}
  \begin{itemize}
    \item What kind of eye diseases are most common?
    \item Find some suitable tasks.
	\item Are there simulators like this available?
    \item How to implement custom camera projections in Unity?
    \item How to interface with the Oculus Rift?
  \end{itemize}

  \item Implementation \Gray{- 1. Jan. '16}
  \begin{itemize}
    \item Create scenario/level 1
    \item Implement the first disease
    \item Implement the other diseases 
    \item Create another scenario
    \item Create menu
    \item Integrate Oculus Rift
  \end{itemize}

  \item Testing \Gray{- 11. Jan. '16}
  \item Report \Gray{- 18. Jan. '16}
  \item Presentation \Gray{- 25. Jan. '16}

\end{itemize}

\section{The human eye}
\begin{table}
    \centering
    \begin{tabular}{ll}
        Diopters                & 59-60 dpt
        Focal length            & 17mm/22-24mm\footnote{See \cite{eye-focal, eyeascamera}} \\
        Pupil diameter          & 2mm - 7/8mm (contracted - dilated)\footnote{See \cite{eyeascamera}} \\
        Cornea-Retina distance  & 17mm/25mm\footnote{See \cite{eyeascamera}} \\
        f-stop                  & ~f/3.2 or f/3.5\footnote{See \cite{eyeascamera}} \\
        Cone of visual attention& ~55º \\
        № Pixels                & 130 mega pix (6 mio cones, 124 rods) (compared to 24 mega pix) \\
        Macula                  & 6mm radius, 150000 px/mm²\\
        Fovea (inside macula)   & color only vision, densly packed with cones \\
    \end{tabular}
    \caption{Properties of the human eye}
    \label{tab:eyeproperties}
\end{table}

Change in focal length yields zoom. Big focal length = big zoom, small focal length = wide angle\\
Moving the fixed focal-length lense away and towards the canvas focuses. \\
Given the cornea-retina distance of 24mm, the focal-length range of the human eye can be estimated by [20,69mm - 24mm] for a distance range from 15cm to ∞ (see research/FocalLengthDiopters.ods) using the formula
\begin{equation}
    \frac{1}{d_p} + \frac{1}{d_I} = \frac{1}{f}.
\end{equation}
In order to simulate Myopia/Hyopia we calculate the required focal length for the altered retinal distance based on
\begin{equation}
    f = \frac{d_p d_I}{d_p + d_I}
\end{equation}
and truncate the value to the 
%
\subsection{Color blindness}
\subsection{Glaucoma}
\subsection{Myopia and Hyperopia}
\begin{description}
\item[Axial Myopia]
    Increase in the eye's axial length.\\
    Most common one!
\item[Refractive Myopia]
    \begin{description}
    \item[Curvature myopia]
    Execssive or increased curvature of one or more of the refractive surfaces of the eye. Mostly the cornea.
    \item[Index myopia]
    Variation of the index of reflection of one or more of the ocular media.
    \end{description}
\item[Nocturnal myopia]
\end{description}

For a reference on real-time focal blur/depth of field see \cite{gpugems-DoF, gpugems3-DoF}.

\todo{}
\printbibliography

\balancecolumns

\end{document}
