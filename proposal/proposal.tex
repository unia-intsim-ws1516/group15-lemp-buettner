\documentclass{acm_proc_article-sp}

\usepackage[T1]{fontenc}
\usepackage{polyglossia}
\setdefaultlanguage{english}

\usepackage{fontspec}
\usepackage{xltxtra}
\usepackage{libertine}

\usepackage{xcolor}
\usepackage{blindtext}

\usepackage[backend=biber, style=alphabetic]{biblatex}
\bibliography{literatur.bib}


\begin{document}


\title{
Interactive Simulation WS 15/16\\ %
Project Proposal
}
\subtitle{EYES - Exchange Your Vision Simulator}
\numberofauthors{2}
\author{
% 1st. author
\alignauthor
Sebastian Lemp\\
%       \affaddr{Street, House}\\
%       \affaddr{PLZ City}\\
%       \affaddr{Country}\\
%       \email{sebastian.lemp@student.uni-augsburg.de}
% 2nd. author
\alignauthor
Stefan Büttner\\
%       \affaddr{Street, House}\\
%       \affaddr{PLZ City}\\
%       \affaddr{Country}\\
%       \email{stefan.buettner@student.uni-augsburg.de}
}
%\additionalauthors{Additional Authors}

% The date is actually not used in the acm template
\date{University of Augsburg, \today}

% Not neede for our purposes
%\terms{Terms}
%\keywords{Keyword 1, Keyword 2}
%% A category with the (minimum) three required fields
%\category{H.4}{Information Systems Applications}{Miscellaneous}
%%A category including the fourth, optional field follows...
%\category{D.2.8}{Software Engineering}{Metrics}[complexity measures, performance measures]

%% For the ACM ToG format
%\acmformat{ACMFormat}
%\acmVolume{Vol.}
%\acmNumber{Nr.}
%\acmYear{YYYY}
%\acmMonth{MM}
%\acmArticleNum{XXX}
%\doi{DOI}


\maketitle
%\begin{abstract}
%\end{abstract}

\section{Introduction/Motivation}
+ Convey understanding of other optical perception than the human eye.

\section{Concept}
\begin{itemize}
\item At least 2 other visual systems to choose from
\item One or more tasks to solve using those available systems.
\item Model the visual systems as realistically as possible.
\item Use oculus rift to address each eye individually → different modes:
  \begin{itemize}
  \item Both human eyes see the same images to have a flat, monitor like view
  \item For binocular systems: One-to-one mapping of the eyes
        For multiocular systemo: Map them somehow to the two human eyes
  \end{itemize}
\end{itemize}

\subsection{User Experience}

\section{Project Requirements}
\begin{description}
\item[Science]
- Optical apparatus and movement of different species, e.g.
    Flies, Spiders, Wasps, Bees, Bats (sonic perception?),
    fish (deep sea?), birds, simple cellular organisms
- Mapping of the perceived spectra of the different species to the human
  perceivable spectrum (papers on this topic)
- Mapping of the layout of one system to the human visual system
See \cite{insectvision}

\item[Gamification]
- Search targets with different eyes. Different species have different
  \"standard\" activities.
  Implement one/two/... which can be solved with any vision system.
  I.e. All n vision systems can be chosen in all m species specific activities,
  or to put it differently: Equip character with another vision system.
  E.g. Ant foraging: search food and return to base as fast as possible.
- Possibility to create your own vision system by configuring n eyes(cameras)
  and attach them to the model (ant, spider, whatever) and choose a layout
  for the 2D-screen / oculus.
  This lets the user test and develop better, task specific visual systems.
  Of course: save/load
- Species-task specific companions/enemies which need to be simulated

\item[Complexity]
- In combination with the oculus rift: visual complexity in
  processing/perceiving the environment.
- learning about the visual systems of other species
- Dunno about model complexity, yet.

\item[Aesthetics]
\end{description}

\section{Timeline}

\printbibliography

\end{document}
