\documentclass[a4paper, 11pt]{scrartcl}

\usepackage[T1]{fontenc}
\usepackage{polyglossia}
\setdefaultlanguage{german}
\usepackage{csquotes}

\usepackage{fontspec}
\usepackage{xltxtra}
\usepackage{libertine}

\usepackage[usenames, dvipsnames, svgnames, table]{xcolor}
\usepackage{graphicx}
\graphicspath{{./img/}}

\usepackage[backend=biber, style=alphabetic]{biblatex}
\bibliography{literatur.bib}

\usepackage{subcaption}
\usepackage{fancyref}
\usepackage{enumerate}

\usepackage[%
unicode=true,%
colorlinks=true,%
linkcolor=black,%
urlcolor=MidnightBlue,%
citecolor=black,%
filecolor=black%
]
{hyperref}

\newcommand{\dateversion}{Augsburg, \today,  Version: 0.0.0}
\newcommand{\authors}{Sebastian Lemp, Stefan Büttner}
\newcommand{\documenttitle}{Interactive Simulation\\Projekt: EYES}

\title{\documenttitle}
\author{\authors}
\date{\dateversion}

\usepackage{scrpage2}

\clearscrheadfoot
\setkomafont{pageheadfoot}{ %
    \bfseries\large}

\ihead[]{\documenttitle} %
\ohead[]{\hfill {\large\scshape \thepage}} %
\ifoot[]{\footnotesize \authors} %
\ofoot[]{\footnotesize %
\dateversion} %
\pagestyle{scrheadings}

\newcommand{\todo}[1]{\textcolor{Red}{\textbf{ToDo: }#1}}
\newcommand{\fertig}{\color{ForestGreen}}
\newcommand{\unwichtig}{\color{Gray}}
\newcommand{\entwickler}[1]{\textcolor{BurntOrange}{(#1)}}
\newcommand{\Stefan}{\entwickler{Stefan}}
\newcommand{\Sebastian}{\entwickler{Sebastian}}


\begin{document}

\maketitle

{\bfseries Legende:}
\begin{itemize}
    \item {\fertig Projektteil implementiert}
    \item {Projektteil noch nicht bearbeitet}
    \item {\unwichtig Projektteil wurde, während des Projkts, als weniger
    wichtig uneingestuft und kann später implementiert oder weggelassen werden.}
\end{itemize}


\section{Timeline}
Bis Mo, 7. Dez. 2015:
\begin{enumerate}
    {\fertig
    \item Bauen des Explorationslevels mit Hilfe des Supermarket Assets. \Sebastian
    \begin{itemize}
        \item Geometrie auslegen (Regale, Produkte, Eingangsbereich, Kassen, Decke, Lichter, etc.
        \item Startpunkt festlegen
        \item First-Person Steuerung implementieren
    \end{itemize}
    }
    \item Glaucoma als Asset, das an die Kamera angeheftet werden kann, implementieren \Sebastian
    {\fertig \item Farbblindheit als Asset, das an die Kamera angeheftet werden kann, implementieren \Stefan}
    \item Dialog zum ein- und abschalten von Krankheiten. Platzhalter für spezialisierten Konfigurationsdialog der jeweiligen Krankheit. \Stefan
\end{enumerate}
Bis Mo, 14. Dez. 2015:
\begin{enumerate}
    \item Alle(!) Produkte können eingesammelt werden und werden in einem "Warenkorb" gespeichert. \Sebastian
    \item Verfeinern des Explorationslevels.
    \begin{itemize}
        \item Kollisionen mit Regalen/Kassen/... implementieren \Sebastian
    \end{itemize}
    \item Einkaufszettel anzeigen \Stefan
    \item Verbleibende Zeit zählen und anzeigen \Stefan
    \item Information über aktuelle Krankheit und deren Schwere anzeigen \Stefan
\end{enumerate}
Bis Mo, 21. Dez. 2015:
\begin{enumerate}
    \item Einstellungsdialog für Glaucoma \Sebastian
    \item Einstellungsdialog für Farbblindheit \Stefan
    \item Myopia / Hyperopia als Asset, das an die Kamera angeheftet werden kann, implementieren \Stefan
    \item Einstellungsdialog für Myopia/Hyperopia \Stefan
    \item Cataract als Asset, das an die Kamera angeheftet werden kann, implementieren \Sebastian
    \item Einstellungsdialog für Cataract \Sebastian
\end{enumerate}
Bis Mo, 4. Jan. 2016:
\begin{enumerate}
    \item Level 1 implementieren \Stefan
    \item Punkteübersicht nach Levelende implementieren \Sebastian
    \item Hauptmenü implementieren. \Stefan Links zu:
        \begin{itemize}
        \item Explorationslevel
        \item Punktelevel
        \item Informationen über die Krankheiten
        \item Ende
    \end{itemize}
\end{enumerate}
Bis Mo, 11. Jan. 2016:
\begin{enumerate}
    \item Level 2 implementieren \Sebastian
    \item Level 3 implementieren \Stefan oder \Sebastian
    \item Informationsmenü für Glaucoma \Sebastian
    \item Informationsmenü für Cataract \Sebastian
    \item Informationsmenü für Farbblindheit \Stefan
    \item Informationsmenü für Myopia/Hyperopia \Stefan
\end{enumerate}
% Sebastian 11
% Stefan 13


\section{Ideensammlung}
\subsection{Level: Supermarkt}
Dies kann in einem Dummylevel (der danach vermutlich dem
\emph{Explorationslevel} umgebaut wird) implementiert und getestet werden.
\begin{itemize}
    \item Bauen des Levels.
    \item Objekte können aufgesammelt werden
    \item Kollisionen (z.B. mit Regale, ...)
    \item Jedes Level hat eine oder mehrere bestimmte Krankheiten welche mit festgesetzten Parameter voreingestellt sind.
    \item Je nach Level ändern sich die Krankheit(en) und/oder die eingestellten Parameter.
    \item Der Spieler muss einen Einkaufszettel abarbeiten.
    \item Start ist am Eingang
    \item Ziel ist an der Kasse.
    \item Punktevergabe wie viele Produkte vom Einkaufszettel in der gegebenen Zeit erreicht worden sind, Bonuspunkte für übrig gebliebene Zeit, Abzug für fehlende/falsche Produkte. Mindestpunktzahl zum Erreichen des nächsten levels.
\end{itemize}
Die verschiedenen Schwierigkeitsgrade für die einzelnen Level werden durch die Krankheiten, deren Parameter, den Einkaufszettel, das Marktlayout und das Zeitlimit bestimmt.


Hier sollten die Ziele, also was man mit dem Level zeigen will sowie deren Realisierung beschreiben werden.
\begin{description}
    \item[Explorationslevel]
    Ein Level in dem man keine Aufgaben erledigen muss aber verschiedene Krankheiten testen kann.
    Dh. man kann zur Laufzeit Krankheiten "zu- und abschalten" und diese auch zur Laufzeit dynamisch konfigurieren. → GUI
    \item[Level 1]
\end{description}


\subsection{GUI}
\begin{itemize}
    \item Zeitanzeige
    \item Einkaufszettel
    \item Einstellungsdialog für jede Krankheit → Parameter der Krankheit identifizieren
\end{itemize}


\subsection{Krankheiten}
\begin{itemize}
    \item Glaucoma \Sebastian \\
    \todo{Parameters}
    \item Cataracts \Sebastian \\
    \todo{Parameters}
    \item Diabethic Retinopathy \Sebastian \\
    \todo{Parameters}
    \item Colorblindness \Stefan \\
    \begin{tabular}{ll}
        Typ & Protanomaly/Deuteranomaly/Tritanomaly \\
        Severity & 0 - 1 \\
        Cone Sensitivity Curves & 2D Grafen \\
    \end{tabular}
    \item Myopia / Hyperopia \Stefan \\
    \todo{Parameters}
\end{itemize}



\subsection{Gameplay}
\begin{itemize}
    \item Ein Bezahl-/Münzensuch Minispiel wäre nice :)
    \item Soll des Spieler in die Rolle des Kassierers schlüpfen können?
    Mögliche Aufgaben wären bspw.:
    \begin{itemize}
        \item Rückgeld geben → Münzen raussuchen
        \item Falls der Barcode nicht gelesen werden kann muss die Ware
        in einer Liste nachgeschlagen werden.
        \item Ausweise bei Alkoholkauf lesen (ausländische Ausweise?)
        \item Kunden klauen, das ist vllt schwieriger zu erkennen?
    \end{itemize}
\end{itemize}

\subsection{Gamedesign}
\begin{enumerate}
    \item Im Menü ggf. Infos über die einzelnen Krankheiten.
    \item Andere Leute die einkaufen, um den Weg zu versperren etc.
    \begin{itemize}
        \item 3+ Modelle
        \item Animationen
        \begin{itemize}
            \item Laufen
            \item Produkte einsammeln/aufheben/einstecken
            \item Zahlen
        \end{itemize}
        \item Pfadplanung
        \item Abläufe skripten
    \end{itemize}
    \item Kassierer
\end{enumerate}


\subsection{Sonstiges}
\begin{enumerate}
    \item Realistische Verpackungen (gibt es Texturen online?)
\end{enumerate}


\end{document}
