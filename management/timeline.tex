\documentclass[a4paper, 11pt]{scrartcl}

\usepackage[T1]{fontenc}
\usepackage{polyglossia}
\setdefaultlanguage{english}
\usepackage{csquotes}

\usepackage{fontspec}
\usepackage{xltxtra}
\usepackage{libertine}

\usepackage[usenames, dvipsnames, svgnames, table]{xcolor}
\usepackage{graphicx}
\graphicspath{{./img/}}

\usepackage[backend=biber, style=alphabetic]{biblatex}
\bibliography{literatur.bib}

\usepackage{subcaption}
\usepackage{fancyref}
\usepackage{enumerate}

\usepackage[%
unicode=true,%
colorlinks=true,%
linkcolor=black,%
urlcolor=MidnightBlue,%
citecolor=black,%
filecolor=black%
]
{hyperref}

\newcommand{\dateversion}{Augsburg, \today,  Version: 0.0.0}
\newcommand{\authors}{Sebastian Lemp, Stefan Büttner}
\newcommand{\documenttitle}{Interactive Simulation\\Projekt: EYES}

\title{\documenttitle}
\author{\authors}
\date{\dateversion}

\usepackage{scrpage2}

\clearscrheadfoot
\setkomafont{pageheadfoot}{ %
    \bfseries\large}

\ihead[]{\documenttitle} %
\ohead[]{\hfill {\large\scshape \thepage}} %
\ifoot[]{\footnotesize \authors} % 
\ofoot[]{\footnotesize % 
\dateversion} %
\pagestyle{scrheadings}

\newcommand{\todo}[1]{\textcolor{Red}{\textbf{ToDo: }#1}}
\newcommand{\fertig}{\color{ForestGreen}}
\newcommand{\unwichtig}{\color{Gray}}
\newcommand{\entwickler}[1]{\textcolor{BurntOrange}{(#1)}}
\newcommand{\Stefan}{\entwickler{Stefan}}
\newcommand{\Sebastian}{\entwickler{Sebastian}}


\begin{document}

\maketitle

{\bfseries Legende:}
\begin{itemize}
    \item {\fertig Projektteil implementiert}
    \item {Projektteil noch nicht bearbeitet}
    \item {\unwichtig Projektteil wurde, während des Projkts, als weniger
    wichtig uneingestuft und kann später implementiert oder weggelassen werden.}
\end{itemize}

\section{Unabdingbare Anforderungen}
\subsection{Supermarkt - Allgemeine Anforderungen}
Dies kann in einem Dummylevel, der danach vermutlich zu \emph{Level 1}
oder dem \emph{Explorationslevel} umgebaut wird implementiert und
getestet werden.
\begin{itemize}
    \item Objekte können aufgesammelt werden
    \item Kollisionen (z.B. mit Regalen, ...)
\end{itemize}
%
%
\begin{description}
    \item[Explorationslevel]
    Ein Level in dem man keine Aufgaben erledigen muss aber verschiedene
    Krankheiten testen kann. Dh. man kann zur Laufzeit Krankheiten
    "zu- und abschalten" und diese auch zur Laufzeit dynamisch
    konfigurieren. → GUI
    \item[Level 1]
    Hier sollten die Ziele, also was man mit dem Level zeigen will sowie
    deren Realisierung beschreiben werden.\\
    \begin{itemize}
        \item Man kann zu Spielbeginn die Krankheiten selbst
        wählen/konfigurieren oder den Zufall entscheiden lassen.
        \item Der Spieler muss einen Einkaufszettel abarbeiten.
        \item Start ist am Eingang
        \item Ziel ist an der Kasse.
        \item Punktevergabe wie viele Produkte vom Einkaufszettel in der gegebenen Zeit erreicht worden sind,             Bonuspunkte für übrig gebliebene Zeit, Abzug für fehlende/falsche Produkte. Mindestpunktzahl zum erreichen         des nächsten levels.
        \item verschiedene Schwirigkeiten für die einzelnen level???
    \end{itemize}
\end{description}


\subsection{GUI}
\begin{itemize}
    \item Zeitanzeige
    \item Einkaufszettel
    \item Einstellungsdialog für jede Krankheit → Parameter der Krankheit
    identifizieren
    \item Einstellungsdialog für jede Krankheit.
\end{itemize}


\subsection{Krankheiten}
\begin{itemize}
    \item Glaucoma \Sebastian \\
    \todo{Parameters}
    \item Cataracts \Sebastian \\
    \todo{Parameters}
    \item Diabethic Retinopathy \Sebastian \\
    \todo{Parameters}
    \item Colorblindness \Stefan \\
    \begin{tabular}{ll}
        Typ & Protanomaly/Deuteranomaly/Tritanomaly \\
        Severity & 0 - 1 \\
        Cone Sensitivity Curves & 2D Grafen \\
    \end{tabular}
    \item Myopia / Hyperopia \Stefan \\
    \todo{Parameters}
\end{itemize}



\section{Optionale Anforderungen}
\begin{enumerate}
    \item Realistische Verpackungen (gibt es Texturen online?)
    \item Andere Leute die einkaufen, um den Weg zu versperren etc.
    \begin{itemize}
        \item 3+ Modelle
        \item Animationen
        \begin{itemize}
            \item Laufen
            \item Produkte einsammeln/aufheben/einstecken
            \item Zahlen
        \end{itemize}
        \item Pfadplanung
        \item Abläufe skripten
    \end{itemize}
    \item Kassierer
\end{enumerate}



\section{Ideensammlung}
\subsection{Gameplay}
\begin{itemize}
    \item Ein Bezahl-/Münzensuch Minispiel wäre nice :)
    \item Soll des Spieler in die Rolle des Kassierers schlüpfen können?
    Mögliche Aufgaben wären bspw.:
    \begin{itemize}
        \item Rückgeld geben → Münzen raussuchen
        \item Falls der Barcode nicht gelesen werden kann muss die Ware
        in einer Liste nachgeschlagen werden.
        \item Ausweise bei Alkoholkauf lesen (ausländische Ausweise?)
        \item Kunden klauen, das ist vllt schwieriger zu erkennen?
    \end{itemize}
\end{itemize}

\subsection{Gamedesign}
\begin{enumerate}
    \item Im Menü ggf. Infos über die einzelnen Krankheiten.
\end{enumerate}


\subsection{Sec 3}

\end{document}
